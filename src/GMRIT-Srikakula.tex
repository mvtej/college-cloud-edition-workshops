% Created 2017-01-25 Wed 10:40
\documentclass[11pt]{article}
\usepackage[utf8]{inputenc}
\usepackage[T1]{fontenc}
\usepackage{fixltx2e}
\usepackage{graphicx}
\usepackage{longtable}
\usepackage{float}
\usepackage{wrapfig}
\usepackage{rotating}
\usepackage[normalem]{ulem}
\usepackage{amsmath}
\usepackage{textcomp}
\usepackage{marvosym}
\usepackage{wasysym}
\usepackage{amssymb}
\usepackage{hyperref}
\tolerance=1000
\author{VLEAD}
\date{\today}
\title{Pilot Workshop at GMRIT, Srikakulam}
\hypersetup{
  pdfkeywords={},
  pdfsubject={},
  pdfcreator={Emacs 24.5.1 (Org mode 8.2.10)}}
\begin{document}

\maketitle
\tableofcontents


\section{Objective of the workshop}
\label{sec-1}
\begin{itemize}
\item College Cloud should be able to serve the labs to the students
to perform experiments or to use the labs without the Internet
connection.

\item Conduct the workshop offline using portable-media.

\item Note down the issues if there are any, during the workshop and
report the issues to \texttt{engg@vlabs.ac.in}
\end{itemize}

\section{Introduction on Pilot Test}
\label{sec-2}
\subsection{Introduction}
\label{sec-2-1}
Introduction about Portable-media
\begin{itemize}
\item What is Portable-Media?
\end{itemize}
\subsection{Demo Lab}
\label{sec-2-2}
\begin{itemize}
\item \texttt{Image Processing Lab}
\end{itemize}
\subsection{Limitations}
\label{sec-2-3}
\begin{itemize}
\item 15-20 students in one run
\end{itemize}
\subsection{Number of Students and Workshop duration}
\label{sec-2-4}
Let's say, for example, 20 students per 30 mins, so for 100
students=5 groups, 5*30mins= 150mins= 2 hrs 30 mins
\subsection{How many labs at a time + Which Labs?}
\label{sec-2-5}
\begin{itemize}
\item Two labs at a time 
\begin{itemize}
\item Problem solving (cse04-iiith.vlabs.a.cin)
\item Image Processing (cse19-iiith.vlabs.ac.in)
\end{itemize}
\end{itemize}
\subsection{Feedback collection with reference to portable media}
\label{sec-2-6}
After the workshop, collecting feedbacks and analytics on
portable-media.

\section{Pilot workshop details}
\label{sec-3}
\subsection{9th Jan 2017 - afternoon}
\label{sec-3-1}
Took Xenon server 
Copied college cloud to GMRIT's 500GB SATA HDD
\subsection{10th Jan 2017}
\label{sec-3-2}
\subsubsection{Morning session 10:00AM to 12:00 PM}
\label{sec-3-2-1}
\begin{itemize}
\item Took the machine
\item Attached GMRIT's 500 GB SATS HDD college cloud
\item Connected wifi routers LAN port to the machine
\item Booted the machine
\item Tested one lab with the college cloud - It was working

\item I have taken five IPs from GMRIT network. These ips will be used
for college cloud setup. So that, every one can access virtual
labs over LAN. I have configured college cloud with the five
ips.
\begin{itemize}
\item I have taken client machine to test college cloud in the same
LAN.
\item Open edX platform was opening but lab was opening very slow.
\item Then I went to the other place where workshop is going
on. Open edX platform was not opening
\end{itemize}

\item I thought of using normal computer that is there in workshop lab
\begin{itemize}
\item It has very less system configuration
\begin{itemize}
\item RAM : 4GB ( which is not enough for our college cloud)
\end{itemize}
\end{itemize}
\item They could not bring other machine that has better system
configuration

\item Finally, I used my lap top
\begin{verbatim}
RAM : 8GB ( 4 GB for Open-edX platform, 4GB for host machine)
CPUs:4  ( 2 for Open-edX platform, 2 for Host machine( cluster))
Processor model : Intel(R) Core(TM) i3-5005U CPU @ 2.00GHz
\end{verbatim}
\end{itemize}
\subsubsection{Afternoon Session 2:00PM to 4:30 PM}
\label{sec-3-2-2}
\begin{itemize}
\item Booted Sony 1 TB USB HDD from my laptop
\item Connected to Wifi-router
\item Removed Incoming connection in the lab in switch from where
all desktop machine gets DHCP IPs.

\item Took one LAN cable and established LAN connection between
switch and Wi-Fi router
\item Now all desktop machines are getting dhcp ip from wi-fi router
\item 30 machines are now connected to college cloud
\item Around 30 students attended to the college cloud pilot
test.
\item Used \textbf{Problem Solving} lab to show demo
\item Then, We asked students to perform experiments on that lab
\item After 40 minutes, I have started another lab \textbf{Data structure} (
Students choice). That means there are now two labs running in
college college.
\item Out of 30 students, 26 students were performing experiments on
two labs
\item We asked students to choose any lab from the running two labs
\textbf{Problem solving} and \textbf{Data structures} and perform any
experiments.
\item Final result, Every student was able to perform any experiment
from any lab from the two labs mentioned above without any
loading issue.
\item At around 4:30 PM, we have asked students to submit their
feedback  on college cloud by providing feedback forms(Hard
copy)
\end{itemize}













\subsection{Overall College cloud pilot test}
\label{sec-3-3}
For College Cloud pilot test, I used following system configuration
\begin{verbatim}
RAM : 8GB ( 4 GB for Open-edX platform, 4GB for host machine)
CPUs:4  ( 2 for Open-edX platform, 2 for Host machine( cluster))
Processor model : Intel(R) Core(TM) i3-5005U CPU @ 2.00GHz
\end{verbatim}

\begin{center}
\begin{tabular}{rrll}
\hline
No.of.Students & No.of.Labs & RAM Usage & CPU usage\\
\hline
26(30) & 1 & TBD & TBD\\
\hline
26(30) & 2 & TBD & TBD\\
\hline
\end{tabular}
\end{center}

\subsubsection{Issues while workshop is going on}
\label{sec-3-3-1}
Feedback link was not working on edX platform.

\section{Analysis/observations}
\label{sec-4}
\begin{description}
\item[{Over the LAN}] 50 students can use College Cloud at a time using
the above system configuration for 5 labs. 

If we use
\begin{verbatim}
Processor : Intel(R) Core(TM) i7-3770 CPU @ 3.40GHz   ( Model : intel-db75en)
RAM : 16 GB
SMPS : 600 W
CPUs: 8 four
MotherBoard wattage: 77 W
\end{verbatim}

I am sure, 100-200 students can use College Cloud
for 10-20 labs at a time( May be more).
\item[{Over the Wi-Fi}] \begin{itemize}
\item System configuration 
\begin{verbatim}
RAM : 8GB ( 4 GB for Open-edX platform, 4GB for host machine)
CPUs:4  ( 2 for Open-edX platform, 2 for Host machine( cluster))
Processor model : Intel(R) Core(TM) i3-5005U CPU @ 2.00GHz
\end{verbatim}
Max 15-20 students can use the College Cloud.

\item System Configuration  
\begin{verbatim}
Processor : Intel(R) Core(TM) i7-3770 CPU @ 3.40GHz   ( Model : intel-db75en)
RAM : 16 GB
SMPS : 600 W
CPUs: 8 four
MotherBoard wattage: 77 W
\end{verbatim}
(\textbf{Assumption}) Max 60-80 students can use College Cloud at a
time
\end{itemize}
\end{description}

\section{Post Work shop}
\label{sec-5}
\begin{itemize}
\item Once we reach VLEAD, IIITH, Hyderabad, I have checked why feedback
link was not working?
\item[{Solution}] That was my mistake. I have loaded up different
vagrant box instead of working one. We had 4-5 vagrant
boxes in College cloud.
\end{itemize}

\section{Conclusion}
\label{sec-6}
As we expected, we are successfully completed pilot test using
college cloud.  My observation is, College Cloud should run on LAN
in order to achieve 100-200 users for 10-20 labs at a time. Over the
Wifi only 20-30 students/lab users can use 3-5 labs at a time, it
very low as we know.
% Emacs 24.5.1 (Org mode 8.2.10)
\end{document}
